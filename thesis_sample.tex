\documentclass[a4paper,12pt]{report}

\usepackage[latin2]{inputenc} % vagy latin2 helyett utf8
\usepackage[T1]{fontenc}      % karakterkódolás
\usepackage[english]{babel}   % angol beállítások
\frenchspacing                % helyközök
%\usepackage{times}           % betûtípus
\usepackage{lmodern}          %   vagy inkább ez

\usepackage[margin=2.5cm,left=3.5cm,includeheadfoot]{geometry}
                              % margók
\usepackage{graphicx}         % képekhez
\usepackage{setspace}         % sorköz
\onehalfspacing               % másfeles




\begin{document}

% ------------------------------------------------------------------------------
% Címlap

\begin{titlepage}

\noindent
\parbox[m]{0.2\textwidth}{
%\includegraphics[width=0.2\textwidth]{elte_logo_bw.eps}     % fekete-fehér
 \includegraphics[width=0.2\textwidth]{elte_logo_color.eps} % színes
}
\hfill
\parbox[m]{0.7\textwidth}{
\begin{center}
\begin{large}
\textsc{
E\"otv\"os Lor\'and University\\
\vspace{0.5pc}
Faculty of Informatics\\
\vspace{0.5pc}
Department of Numerical Analysis\\
}
\end{large}
\end{center}
}

\vspace{1pc}
\hrule

\vfill

\begin{center}
{\LARGE Digital Siganl Processing Plugin\\for Multilayered Synthesis}
\end{center}

\vfill

\noindent
\hspace*{0.05\textwidth}
\parbox{0.45\textwidth}{
{\it Supervisor:}
\bigskip

{\Large Viktoria Zs\"ok}
\smallskip

Assistant Lecturer
}
\hfill
\parbox{0.45\textwidth}{
{\it Author:}
\bigskip

{\Large Evan Sitt}
\smallskip

Computer Science BSc
}


\vfill

\begin{center}
{\large {\it Budapest, 2014}}
\end{center}

\end{titlepage}


% ------------------------------------------------------------------------------
% Témabejelentõ

\vspace*{\fill}
\begin{center}
This page should be the original Thesis Topic Declaration.
\end{center}
\vfill
\thispagestyle{empty}
\newpage
\setcounter{page}{1}

% ------------------------------------------------------------------------------
% Tartalomjegyzék

\tableofcontents

% ------------------------------------------------------------------------------


\chapter{Introduction}

The Introduction should summarize the motivation of the topic for the thesis and short summary and plain description of the project.


% ------------------------------------------------------------------------------

\chapter{Background}
% - Digital Synthesis
Digital synthesis is a field in which an audio engineer utilizes digital signal processing techniques to make musical sounds. One of the most prominent early examples were the keyboards manufactured by Yamaha, which employed their proprietary FM Synthesis algorithms back in the 1970s. \\
A frequency synthesizer refers to a synthesizer with circuitry capable of generating a range of frequencies from a single reference frequency input. One typical application of a frequency synthesizer is the use of a Fourier Series to construct waveforms via Additive Synthesis.\\
% - Direct Digital Synthesis
When designing a digital synthesizer, there is a multitude of techniques available to the designer for generating frequencies, including Phase Locked Loop, Digital-Analog Conversion, and Direct Digital Synthesis. This project approaches the problem of digital synthesis by using Direct Digital Synthesis. Direct Digital Synthesis is a technique utilized by modern frequency synthesizers. Direct Digital Synthesis utilizes a singular fixed-frequency reference clock as a stable time basis for the entire system. In contrast to the alternative Phase-Locked Loop system, where the output phase is locked directly to the phase of the input signal, Direct Digital Synthesis holds significant advantages in flexibility, the most prominent being better frequency agility. Frequency agility is defined as the capacity to switch the oscillator between different frequencies without introducing a significant delay in the signal. This advantage is offered as a result of Direct Digital Synthesis using discrete sample operations. Other advantages offered by the Direct Digital Synthesis technique include access to a broader spectrum of frequencies and finer frequency resolution. Additionally, a Direct Digital Synthesis device has the potential to have low overhead and power draw.\\
A Direct Digital Synthesis device typically consists of a few parts: a digital signal input, a phase accumulator, a phase register, a phase to amplitude converter, a digital to analog converter, and finally the reference oscillator.\\
%   - Digital Signal Input
The digital signal input acts as the Tuning Word for the system. The demanded frequency value is provided as an input to the system. Additional information that can be provided at this stage of the system includes the type of waveform to be generated, along with any MIDI instructions for the other signal processing modules further down the signal chain.\\
%   - Phase Accumulator
The Phase Accumulator holds the digital number, which represents the current phase of the Direct Digital Synthesizer. The Phase Accumulator iterates the phase with a value controlled by the Tuning Word input. The frequency and sampling rate make up the Tuning Word. Once the phase stored overflows a specified threshold value, modulus mathematics is used to reset the phase within the range specified. With a higher frequency, the Tuning Word will cause the Phase Accumulator to cycle through the phase range more rapidly, in essence accomplishing the task of converting the frequency input into a phase value. In a way, the Phase Accumulator can be thought of as a modulus counter.\\
%   - Phase Register
%   - Phase to Amplitude Converter
After the Phase Accumulator step, the phase value is then passed to the Phase to Amplitude Converter. The goal of the Phase to Amplitude Converter is to change the phase value to a digital amplitude value, which can then be further processed. This can be accomplished via two different methods: a Phase/Wavetable lookup, or a Waveform Function.\\
%     - Phase/Wavetable Lookup
The Phase/Wavetable Lookup method was developed in the late 1970s as a more efficient method of generating waveforms on low spec hardware devices. The concept relies on the efficiency advantage of accessing values from a stored array over repeated mathematical calculations of sinusoidal functions. While the hardware constraints are no longer an issue for modern digital synthesizers, the technique is still used for its relative flexibility when compared to the use of sinusoidal functions.\\
For the Phase/Wavetable Lookup method, a wavetable is stored within the system. This wavetable is named so since it contains a predefined number of precomputed amplitude values of the desired waveform. Typically this waveform is a basic sine wave, as other waveforms can be achieved via Fourier series reconstruction, which relies on additive and subtractive operations rather than trigonometric functions. Additionally, the wavetable is typically stored within a direct-access data structure such as arrays for even better efficiency.\\
For each phase value received from the Phase Accumulator, the Phase to Amplitude converter will use the Phase Register to convert the phase to its respective index value in the wavetable. The amplitude value stored at this index value is retrieved from the wavetable and returned. Since the index value generated typically falls between two integer values, usually some form of numerical interpolation is used. The final amplitude value will then be passed on to the rest of the digital signal processing chain.\\
The Phase/Wavetable Lookup method holds significant advantages that contribute to its popularity and continued usage. The efficiency of constant time complexity generation of each sample combined with the flexibility to store any waveform allows for nearly unbeatable processing speed for waveforms.\\
%     - Waveform function
The other method for Phase to Amplitude Conversion is the use of a Waveform Function. Typically Waveform Functions are used whenever the waveform to be generated can be computed with simple additive and multiplicative computations. Examples of waveforms that typically fit this are the sawtooth and triangle waveform types, as they contain linear sections. Another use case for Waveform Functions is in the case of the square and noise waveform types. In both cases, the resulting amplitude values can be "computed" via without involving any mathematics.
An advantage of using a Waveform Function to convert the Phase to Amplitude is that there is generally no need for interpolation, which results in more accurate amplitude value results. However, as there is typically some form of calculation involved this method is generally less efficient than the Phase/Wavetable lookup method by a small margin.\\
% - Additive/Subtractive synthesis
After the Phase to Amplitude conversion, the resulting amplitude value is passed down for Additive or Subtractive synthesis. Additive and Subtractive synthesis are exactly what their names imply; they involve the use of additive/subtractive operations between waveforms to create a new waveform. Applied in a systematic method, distinct types of waveforms can be formed.\\
One method is to use the Fourier series approximation of a waveform. For example, the square wave can be generated via a weighted summation of harmonics of a sine wave. Starting with a basic sine wave at a base frequency, we add to this sine waveform each odd harmonic. A harmonic of a waveform is the waveform with a frequency at an integer multiple of the base frequency of the original waveform. The amplitude of each odd harmonic is multiplied by the reciprocal of the harmonic's frequency multiplier in relation to the base frequency.\\
Another method is to use additive and subtractive synthesis to replicate sounds found in real life, a technique called physical modeling. For example, a piano's sound can be characterized by analyzing a spectral graph of its frequencies. Afterward, this data can be used to generate a set of base frequency multipliers and amplitude multipliers. An Additive/Subtractive synthesizer will then use this set of frequency and amplitude multiplier values to replicate the sound of the piano with high accuracy. This use of Additive and Subtractive synthesis for physical modeling is the technique used and popularized by the Yamaha synthesizers in the 1970s.\\
% - LFO
After the Additive and Subtractive synthesis processes generate a signal for the desired waveform, this is then passed on to the Low-Frequency Oscillator Controlled Modulation processes. Usually referred to as LFO's, these are modulation processes where the value being modulated is modified by a value produced by an oscillator. This oscillator typically generates values using a basic sine wave. However, any waveform can be used, resulting in particularly interesting sound characteristics.\\
Of particular interest are the two most commonly used LFO's: the amplitude LFO and the frequency LFO. Both will modulate their respective parameters using the value from the low-frequency oscillator. An amplitude LFO will reproduce the effect of a tremolo, while the frequency LFO will, on the other hand, reproduce the effect of a vibrato. LFO's can be further controlled by processing the signal from the control oscillators in the same manner as a normal oscillator, although in general, this processing is typically limited to just an Envelope. \\
% - Envelope
After the LFO processes the signal, it is then sent to the Envelope. An Envelope is a generalized term describing a function or set of values that control the modulation value of a certain characteristic of a digital signal. Envelopes can be applied to many different types of modulation, including amplitude modulation and frequency modulation. An envelope can also be applied to the frequency and amplitude multiplier values used within Additive and Subtractive synthesis for further refined sounds.\\
%   - ADSR
The majority of digital synthesizers employ an Amplitude Modulation Envelope. These types of envelopes focus on modifying the amplitude of the sound over time, producing unique characteristics in sound, which can cause a synthesizer to produce the sound of a guitar vs. a violin. The simplest and most widely used envelope type is the ADSR envelope. ADSR stands for Attack-Decay-Sustain-Release, as these are the four parameters that the Envelope modifies.\\
The Attack phase starts from the noteOn MIDI instruction and determines the amount of time required for the amplitude of the sound to transition from silence to its full value as determined by the MIDI velocity. After the Attack phase completes, the Decay phase begins, which acts in opposition to the Attack phase by determining the amount of time required for the amplitude to decrease down to its Sustain level. The Sustain level controls what percentage of the maximum amplitude the sound should be allowed to reach for the duration of the note. And finally, upon receiving a noteOff MIDI instruction, the Release phase begins, which determines the amount of time required for the sound amplitude to decrease back to silence.\\
These four values can drastically shape the character of a synthesizer's sound. For example, let us say we have produced a proper approximation of a string instrument sound via Additive and Subtractive synthesis. Using an Envelope with a long Attack phase, short Decay phase, high Sustain level, and long Release phase could produce a sound akin to a Cello played legato. On the other hand, using an Envelope with an almost instantaneous Attack phase, a minuscule Decay phase, low Sustain level, and a medium-length Release phase could produce a sound closer to that of an Acoustic Guitar. With proper control of the Envelope parameters, it is possible to replicate many different instruments and styles of playing.\\
% - Filters
%   - Low Pass
%   - High Pass
%   - Band Pass
% - Time-Based Delays
%   - Delay Chain
%   - Flanger
% - MIDI
%   - Handling the buffer
% - PCM Wav
%   - Handling the buffer
% - VST3 Standard
%   - System workflow
% - JUCE Framework
%   - Brief Overview

% 7 - 10 pages Background
% May 1st Deadline, 45 days left.
% ------------------------------------------------------------------------------

\chapter{User Documentation}

% The User Documentation (or User's Manual) should contain
% \begin{itemize}
% \item the short description of the solved problem,
% \item the short summary of the used methods and tools,
% \item every required information about the usage of the software.
% \end{itemize}

\begin{itemize}
    \item Installation
    \item Initialization
    \item Workflow Integration
    \item Module Details
    \item Testing details
\end{itemize}

% ------------------------------------------------------------------------------


\chapter{Developer Documentation}

% The Developer Documentation (or Developer's Manual) should contain
% \begin{itemize}
% \item the detailed specification of the problem,
% \item the detailed description of the used methods, the definitions of the used notions,
% \item the description of the logical and physical structure of the software (data structures, databases, modules etc.),
% \item the testing plan and the results of the tests.
% \end{itemize}

\begin{itemize}
    \item Short Background of Task
    \item Implementation of Task
    \item Discussion of implementation
    \item Results - Latency, Under/Overruns, Distortion/Clipping
\end{itemize}


% ------------------------------------------------------------------------------

\chapter{Appendix}

% ------------------------------------------------------------------------------

\begin{thebibliography}{9}
\bibitem{v} Somebody, something.
\end{thebibliography}

\end{document}
