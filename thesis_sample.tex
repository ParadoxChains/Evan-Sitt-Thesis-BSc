\documentclass[a4paper,12pt]{report}

\usepackage[latin2]{inputenc} % vagy latin2 helyett utf8
\usepackage[T1]{fontenc}      % karakterkódolás
\usepackage[english]{babel}   % angol beállítások
\frenchspacing                % helyközök
%\usepackage{times}           % betûtípus
\usepackage{lmodern}          %   vagy inkább ez

\usepackage[margin=2.5cm,left=3.5cm,includeheadfoot]{geometry}
                              % margók
\usepackage{graphicx}         % képekhez
\usepackage{setspace}         % sorköz
\onehalfspacing               % másfeles




\begin{document}

% ------------------------------------------------------------------------------
% Címlap

\begin{titlepage}

\noindent
\parbox[m]{0.2\textwidth}{
%\includegraphics[width=0.2\textwidth]{elte_logo_bw.eps}     % fekete-fehér
 \includegraphics[width=0.2\textwidth]{elte_logo_color.eps} % színes
}
\hfill
\parbox[m]{0.7\textwidth}{
\begin{center}
\begin{large}
\textsc{
E\"otv\"os Lor\'and University\\
\vspace{0.5pc}
Faculty of Informatics\\
\vspace{0.5pc}
Department of Numerical Analysis\\
}
\end{large}
\end{center}
}

\vspace{1pc}
\hrule

\vfill

\begin{center}
{\LARGE Digital Siganl Processing Plugin\\for Multilayered Synthesis}
\end{center}

\vfill

\noindent
\hspace*{0.05\textwidth}
\parbox{0.45\textwidth}{
{\it Supervisor:}
\bigskip

{\Large Viktoria Zs\"ok}
\smallskip

Assistant Lecturer
}
\hfill
\parbox{0.45\textwidth}{
{\it Author:}
\bigskip

{\Large Evan Sitt}
\smallskip

Computer Science BSc
}


\vfill

\begin{center}
{\large {\it Budapest, 2014}}
\end{center}

\end{titlepage}


% ------------------------------------------------------------------------------
% Témabejelentõ

\vspace*{\fill}
\begin{center}
This page should be the original Thesis Topic Declaration.
\end{center}
\vfill
\thispagestyle{empty}
\newpage
\setcounter{page}{1}

% ------------------------------------------------------------------------------
% Tartalomjegyzék

\tableofcontents

% ------------------------------------------------------------------------------


\chapter{Introduction}

The Introduction should summarize the motivation of the topic for the thesis and short summary and plain description of the project.


% ------------------------------------------------------------------------------

\chapter{Background}
Digital synthesis 
% ------------------------------------------------------------------------------

\chapter{User Documentation}

% The User Documentation (or User's Manual) should contain
% \begin{itemize}
% \item the short description of the solved problem,
% \item the short summary of the used methods and tools,
% \item every required information about the usage of the software.
% \end{itemize}

\begin{itemize}
    \item Installation
    \item Initialization
    \item Workflow Integration
    \item Module Details
    \item Testing details
\end{itemize}

% ------------------------------------------------------------------------------


\chapter{Developer Documentation}

% The Developer Documentation (or Developer's Manual) should contain
% \begin{itemize}
% \item the detailed specification of the problem,
% \item the detailed description of the used methods, the definitions of the used notions,
% \item the description of the logical and physical structure of the software (data structures, databases, modules etc.),
% \item the testing plan and the results of the tests.
% \end{itemize}

\begin{itemize}
    \item Short Background of Task
    \item Implementation of Task
    \item Discussion of implementation
    \item Results - Latency, Under/Overruns, Distortion/Clipping
\end{itemize}


% ------------------------------------------------------------------------------

\chapter{Appendix}

% ------------------------------------------------------------------------------

\begin{thebibliography}{9}
\bibitem{v} Somebody, something.
\end{thebibliography}

\end{document}
